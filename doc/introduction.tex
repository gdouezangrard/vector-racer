\section*{Introduction}
\textit{Vector Racer} est originellement un jeu de course de voitures, que l'on peut déplacer sur des cases d'une grille. Le jeu gère les notions de vitesses et d'accélération des voitures, si bien qu'il devient difficile de tourner rapidement avec une vitesse trop élevée. L'objectif de ce jeu est d'arriver en tête de la course tout en évitant les obstacles comme les murs par exemple.

Comme toute course, le joueur cherchera à minimiser son temps de parcours et de fait, à maximiser la vitesse tout en minimisant la distance parcourue. On comprend déjà que notre algorithme de résolution va chercher un parcours qui va correspondre en tout point à une optimisation à la fois de la distance parcourue et de la vitesse.

Nous allons donc poser dans un premier temps les problèmes à résoudre ainsi que le vocabulaire que nous allons employer pour caractériser le jeu. Nous verrons ensuite comment nous avons implémenté les types abstraits de données qui apparaissent dans les problèmes évoqués ci-dessus puis nous terminerons par présenter l'algorithme de résolution et l'ensemble des tests réalisés pour vérifier sa validité.

\subsection*{Capacité du programme réalisé}

Notre programme permet d'afficher, de jouer et de résoudre une carte donnée en paramètre.
La résolution de la carte n'est pas une résolution optimale. Il s'agit d'une solution non triviale de parcours du circuit.